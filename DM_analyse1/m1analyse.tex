%==============================================================
\lilleset{
  formation=Master 1\iere année,
  parcours=parcours Mathématiques,
  annee=2024--2025,
  module=Analyse,
  couleur=teal
}
%==============================================================
% -------------- Autres bibliothèques
\usepackage{dsfont} % pour \mathds{1}

% -------------- Les abréviations pour M1AN
% les ensembles standard
\def\R{\mathbb{R}}
\def\C{\mathbb{C}}
\def\N{\mathbb{N}}
\def\Z{\mathbb{Z}}
\def\Q{\mathbb{Q}}
% les parenthèses et leurs dérivées
\newcommand{\p}[1]{\left(#1\right)}% les parenthèses automatiques
\newcommand{\bp}[1]{\bigl(#1\bigr)}% les parenthèses 'big'
\newcommand{\Bp}[1]{\Bigl(#1\Bigr)}% les parenthèses 'Big'
\newcommand{\Bbp}[1]{\biggl(#1\biggr)}% les parenthèses 'bigg'
\newcommand{\BBp}[1]{\Biggl(#1\Biggr)}% les parenthèses 'Bigg'
\newcommand*{\suite}[2][n]{\left( #2 \right)_{#1}}% exemple \suite{u_n}, \suite[k>0]{u_k}
\newcommand*{\suiteN}[1]{\suite[n\in\mathbb{N}]{#1}}% exemple \suiteN{u_n}
\newcommand*{\abs}[1]{\left\lvert{\ifx\hfuzz#1\hfuzz \,\cdot\,\else#1\fi}\right\rvert}% |.|
\newcommand*{\norm}[1]{\left\lVert{\ifx\hfuzz#1\hfuzz \,\cdot\,\else#1\fi}\right\rVert}% norme
\newcommand*{\norminf}[1]{\norm{#1}_{\infty}}% la norme sup
\newcommand*{\vertiii}[1]{{\left\vert\kern-0.25ex\left\vert\kern-0.25ex\left\vert
  {\ifx\hfuzz#1\hfuzz \,\cdot\,\else#1\fi}
    \right\vert\kern-0.25ex\right\vert\kern-0.25ex\right\vert}}
\let\normop\vertiii% exemple \normop{T}
\newcommand*{\scalprod}[3][]{#1\langle{#2}\kern1pt #1|{#3}#1\rangle}% exemple \scalprod[\big]{A}{B}
\newcommand*{\ensemble}[3][]{#1\{ #2 \;#1|\; #3 #1\}}% exemple \ensemble[\big]{x^2}{x \in \R}
% indication et remarque
\newcommand{\indication}[1]{\emph{Indication : #1}}
\newcommand{\remarque}[1]{\emph{Remarque : #1}}
% la fonction indicatrice
\usepackage{dsfont}
\newcommand*{\ind}{\mathds{1}}% exemple \ind_{[0,1]}
% autre symboles
\DeclareMathOperator{\im}{Im}% exemple \im A
\let\ker\relax\DeclareMathOperator{\ker}{Ker}% exemple \ker A
\DeclareMathOperator{\vect}{vect}% exemple \vect\ensemble{x_n}{n\in\N}
\DeclareMathOperator{\dist}{dist}% exemple \dist(x,M)
\DeclareMathOperator{\id}{Id}% exemple T-\lambda\id
\DeclareMathOperator{\supp}{supp}% le support, exemple \supp \ind_{A} = A
\renewcommand{\Re}{\mathfrak{Re}}% la partie réelle
\newcommand{\dd}{\kern1pt\mathrm{d}}% le «d» du dx, dt, ...

% pour surligner
\sisolutions{
  \usepackage{soul}
  \colorlet{hl}{yellow!35!white}
  \sethlcolor{hl}
  \renewcommand{\hl}[1]{\relax\ifmmode\colorbox{hl}{\ensuremath{#1}}\else\texthl{#1}\fi}
}
